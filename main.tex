\documentclass[11pt, a4paper, english, twoside]{article}

\usepackage[english]{babel}
\usepackage[utf8x]{inputenc}
\usepackage{amsmath, amsthm, amssymb}
\usepackage{mathtools} %for \coloneqq
\usepackage{faktor}
\usepackage{tikz-cd} 
%bibliography
\usepackage[noadjust]{cite}
\usepackage{hyperref}
\usepackage{enumitem}
%font
%\usepackage[euler-digits, euler-hat-accent]{eulervm}
%\usepackage{newpxtext}
%\usepackage{charter}

\theoremstyle{plain}
\newtheorem{theorem}{Theorem}[section]
\newtheorem{corollary}[theorem]{Corollary}
\newtheorem{lemma}[theorem]{Lemma}

\theoremstyle{definition}
\newtheorem{definition}[theorem]{Definition}
\newtheorem{remark}[theorem]{Remark}
\newtheorem{example}[theorem]{Example}
\newtheorem*{acks}{Acknowledgments}

\begin{document}
%ASSUMPTIONS:
%basic algebraic terms like: ring, field, polynomial, algebraic closure, ideals,...

These notes try to give a overview over the various formulations of Hilberts Nullstellensatz, some important applications and different proofs and proof-ideas.

One way to approach the Nullstellensatz, as explained in Taos blogpost \textbf{TODO: cite}, is by studying obstructions to solving a set of simultaneous polyonomial equations over a algebraically closed field $k$.
Say we are given equations of the form $f_1(x) = f_2(x) = \dots = f_l(x) = 0$ with $x = (x_1, \dots, x_n) \in k^n$. If we find polynomials $g_1, \dots, g_l$ such that $f_1g_1 + \dots + f_lg_l = 1$ this equation cannot have a
solution. The weak form of Hilberts Nullstellensatz states that (over algebraically closed fields) this is in fact the only obstruction.

\begin{theorem}[Nullstellensatz, weak form]
    Let $k$ be an algebraically closed field and $f_1,\dots,f_l \in k[x_1,\dots,x_n]$ polyonmials in $n$ variables. 
    Then exactly one of the following statements holds:
    \begin{enumerate}
        \item The system of equations $f_1(x) = \dots =f_l(x) = 0$ has a solution $x \in k^n$.
        \item There exist $g_1, \dots, g_l \in k[x_1,\dots,x_n]$ such that $f_1g_1 + \dots + f_lg_l = 1$.
    \end{enumerate}
\end{theorem}
The way this is used most often, is to conclude the existence of a relation as in the second statement purely from the fact that the $f_i$ admit no simoultaneous zero.
We can restate this in a more algebro-geometric fashion to see a first glimpse of how the Nullstellensatz provides a dictionary between the algebraic and the geometric world. For this we introduce the notion of a 
(classical) affine variety.

\begin{definition}
    Let $k$ be an algebraically closed field and $f_\alpha \in k[x_1,\dots,x_n]$ a collection of polynomials. The \emph{zero set} of the $\{f_\alpha\}$ is defined as
    \[
        Z(\{f_\alpha\}) \coloneqq \{ x \in k^n \,| f_\alpha(x) = 0 \} \subset k^n.
    \]
    Any subset of $k^n$ that is the zero set of some polynomials is called an \emph{affine variety}.
    Given an affine variety $X \subset k^n$ we define its associated ideal as
    \[ 
        I(X) \coloneqq \{f \in k[x_1,\dots,x_n] | f(x) = 0 \text{ for all } x \in X \},
    \]
    and its coordinate ring as
    \[
        A(X) \coloneqq k[x_1,\dots,x_n]/I(X).
    \]
\end{definition}

\begin{remark}
    \label{zeroIdeal}
    $I(X)$ is in fact an ideal: if two polynomials $f$ and $g$ vanish at some point $x$, $f+g$ vanishes as well as $hf$ for some arbitrary polynomial $h$.
    Also, it does not matter if we take the zero-set of some collection of polynomials $\{f_\alpha\}$ or the zero-set of the ideal generated by
    these polynomials. The inclusion $Z((f_\alpha)) \subset Z(\{f_\alpha\})$ is clear since passing to the ideal can only cut out more.
    For the other inclusion note that if $x \in Z(\{f_\alpha\})$, then by the above remark all polynomials in the ideal vanish there as well.
\end{remark}

With these definitions at hand, the reformulated weak Nullstellensatz reads as follows.

\begin{theorem}[weak Nullstellensatz, second formulation]
    Let $k$ be an algebraically closed field, $\mathfrak{a} \subset k[x_1,\dots,x_n]$ an ideal, then $Z(\mathfrak{a})$ is empty if and only if 
    $\mathfrak{a}$ is the unit ideal, i.e.\ all of $k[x_1,\dots,x_n]$.
\end{theorem}

In fact, it is not completely obvious that these two formulations are equivalent, since the ideal $\mathfrak{a}$ might not be finitely generated
and hence we would have to make a statement about infinitely many simoultaneous polynomial equations. That this is not the case is essentially
Hilberts Basis Theorem.

\begin{theorem}[Hilberts Basis Theorem]
    Let $A$ be a noetherian ring (i.e.\ all its ideals are finitely generated), then the polynomial ring $A[x]$ is noetherian as well
\end{theorem}
\begin{proof}[Proof. (AM Thm. 7.5) ]
    Let $\mathfrak{a} \subset A[x]$ be an ideal. Denote by $\mathfrak{b} \subset A$ the ideal generated by all leading coefficients of the
    polynomials in $\mathfrak{a}$. Since $A$ is noetherian, $\mathfrak{b}$ is finitely generated, say $\mathfrak{b} = (a_1,\dots,a_n)$. Thus
    we find polynomials $f_1,\dots,f_n \in \mathfrak{a}$ of the form 
    \[
        f_i = a_ix^{r_i} + \dots
    \]
    These $f_i$ generate an ideal $\mathfrak{a}' \subset \mathfrak{a}$.
    Denote by $r$ the maximum of the degrees $r_i$. Now let $f \in \mathfrak{a}$ be a polynomial of degree greater than or equal to r, i.e.\ 
    $f = bx^n + \dots$ with $n \ge r$. Since the $a_i$ generate $\mathfrak{b}$ we can write 
    \[
        b = \sum_{i=1}^n u_ia_i 
    \]
    with some $u_i \in A$. Hence we can choose $g \in \mathfrak{a}'$ as 
    \[
        g = \sum_{i=1}^n u_ix^{n-r_i}f_i.
    \]
    to get that $f - g$ has strictly smaller degree than $f$. 
    Hence we can write every polynomial in $\mathfrak{a}$ as the sum of some polynomials in $\mathfrak{a}'$ and
    a polynomial in $\mathfrak{a}$ of degree smaller than $r$. Or, in other words, we have a decomposition of $\mathfrak{a}$ as an $A$-module
    \[
        \mathfrak{a} = ( <1,x,\dots,x^{r-1}> \cap \mathfrak{a} ) + \mathfrak{a}'.
    \]
    Since the first summand is a submodule of a finitely generated module over an noetherian ring and $\mathfrak{a}'$ is finitely generated by
    construction, $\mathfrak{a}$ is finitely generated as well. This proves the theorem.
\end{proof}

\begin{corollary}
    If $A$ is a noetherian ring, then $A[x_1,\dots,x_n]$ is noetherian as well.
\end{corollary}

\begin{lemma}
    The two formulations of the weak Nullstellensatz are equivalent.
\end{lemma}
\begin{proof}
    By considering the ideal $\mathfrak{a} \coloneqq (f_1, \dots, f_n)$ and using the fact from remark \ref{zeroIdeal} that 
    $Z(f_1,\dots,f_n) = Z(\mathfrak{a})$ we see that the second formulation implies the first.
    For the converse just note that by hilberts basis theorem every ideal $\mathfrak{a} \subset k[x_1,\dots,x_n]$ is finitely generated and we
    can invoke the first formulation for these finitely many polynomials.
\end{proof}
\end{document}

The first important corollary of the weak Nullstellensatz is the following characterization of maximal ideals in a polynimial ring over an
algebraically closed fields.
\begin{theorem}
    Let $k$ be an algebraically closed field, then the maximal ideals in $k[x_1,\dots,x_n]$ are exactly the ideals of the form
    \[
        (x_1 - a_i, \dots, x_n - a_n)
    \]
    for some $a_i \in k$.
\end{theorem}
\begin{proof}
\end{proof}
